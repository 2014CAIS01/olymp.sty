\gdef\thisproblemauthor{Иван Казменко}
\gdef\thisproblemdeveloper{Иван Казменко}
\begin{problem}{Арифметика на доске}
{arithmetic.in}{arithmetic.out}
{2 секунды (\textsl{3 секунды для Java})}{256 мебибайт}{}

Евгения написала на доске $n$ положительных целых чисел.
Алексей может стирать любые два числа $x$ и $y$,
заменяя их либо на $x + y$, либо на $x \cdot y$, либо на $|x - y|$.
Замены Алексей производит до тех пор, пока не останется одно число.
Какое минимальное число может получиться у Алексея?

\InputFile

В первой строке ввода задано целое число $n$
($1 \le n \le 100\,000$).
Во второй строке заданы через пробел $n$ целых чисел
$a_1$, $a_2$, $\ldots$, $a_n$, которые Евгения изначально написала на доске
($1 \le a_i \le 30$).

\OutputFile

В первой строке выведите минимальное число, которое может получиться у Алексея.

\Examples

\begin{example}
\exmp{
2
1 2
}{%
1
}%
\exmp{
3
1 2 3
}{%
0
}%
\exmp{
4
16 2 3 4
}{%
2
}%
\end{example}

\Explanations

В первом примере можно заменить $1$ и $2$ на $|2 - 1| = 1$.

Во втором примере можно сначала заменить $1$ и $2$ на $1 + 2 = 3$,
а затем получить $|3 - 3| = 0$.

В третьем примере двойку можно получить, например, так:
$2 + 4 = 6$, $3 \cdot 6 = 18$ и $|16 - 18| = 2$.

\end{problem}
