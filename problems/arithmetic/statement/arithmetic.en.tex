\gdef\thisproblemauthor{Ivan Kazmenko}
\gdef\thisproblemdeveloper{Ivan Kazmenko}
\gdef\thisproblemorigin{\texttt{XXXIII} St. Petersburg State University Championship}
\begin{problem}{Arithmetic on a Board}
{arithmetic.in}{arithmetic.out}
{2 seconds (\textsl{3 seconds for Java})}{256 mebibytes}{}

Evgenia wrote $n$ positive integers on a board.
Aleksey can erase any two integers $x$ and $y$, replacing them
by either $x + y$ or $x \cdot y$ or $|x - y|$.
Aleksey makes replacements until there is only one integer left.
What is the minimal integer Aleksey can get after all replacements?

\InputFile

The first line of input contains an integer $n$
($1 \le n \le 100\,000$).
The second line contains a space-separated list of $n$ integers
$a_1$, $a_2$, $\ldots$, $a_n$ which Evgenia initially wrote on the board
($1 \le a_i \le 30$).

\OutputFile

On the first line, print the minimal integer Aleksey can get.

\Examples

\begin{example}
\exmp{
2
1 2
}{%
1
}%
\exmp{
3
1 2 3
}{%
0
}%
\exmp{
4
16 2 3 4
}{%
2
}%
\end{example}

\Explanations

In the first example, you can replace $1$ and $2$ by $|2 - 1| = 1$.

In the second example, you can first replace $1$ and $2$ by $1 + 2 = 3$,
and after that, get $|3 - 3| = 0$.

In the third example, one way to get a two is the following:
$2 + 4 = 6$, $3 \cdot 6 = 18$ and $|16 - 18| = 2$.

\end{problem}
