\begin{problem}{Пример условия задачи}
{\textsl{standard input}}{\textsl{standard output}}
{2 секунды (\textsl{3 секунды для Java})}{256 мебибайт}{}

Здесь написано условие задачи.

\InputFile

В первой строке ввода заданы через пробел два целых числа $n$ и $k$
($1 \le n \le 100\,000$, $1 \le k \le 10$).
Во второй строке заданы через пробел $n$ целых чисел
$a_1$, $a_2$, $\ldots$, $a_n$
($1 \le a_i \le 10^9$).

\OutputFile

В единственной строке выведите одно число "--- ответ на задачу.

\Examples

\begin{example}
\exmp{
2 1
1 2
}{%
1
}%
\exmp{
3 3
1 2 3
}{%
0
}%
\end{example}

\Explanations

В первом примере выведем $1$.

Во втором примере можно сначала подумать, а затем вывести $0$.

\end{problem}
